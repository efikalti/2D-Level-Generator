\chapter{Συμπεράσματα}

\section{Περίληψη}
\par
Για την εξαγωγή των συμπερασμάτων μπορούμε να ξεχωρίσουμε τρεις κύριες κατηγορίες προς αξιολόγηση. Η πρώτη κατηγορία αφορά το σύστημα του PCG και τις αποδόσεις του σε σχέση με τον στόχο για τον οποίο υλοποιήθηκε. Η δεύτερη κατηγορία αφορά τα δύο μοντέλα GAN που αναπτύχθηκαν και η μεταξύ τους σύγκριση με βάση τις μετρήσεις και τα αποτελέσματα που παρήγαγαν. Τέλος, η τρίτη κατηγορία αφορά την συνεργασία των δύο συστημάτων PCG και MLCG.
\par
Έχουμε δύο τρόπους αξιολόγησης των παραπάνω κατηγοριών. Ο πρώτος τρόπος περιλαμβάνει την καταγραφή των αποτελεσμάτων και την χρήση γνωστών μετρικών όπως είναι η ακρίβεια (accuracy) και το μέσο τετραγωνικό σφάλμα (mean square error) για την αξιολόγηση των GAN μοντέλων, καθώς και την χρήση κάποιας ορισμένης από εμάς μετρικής για την αξιολόγηση του PCG συστήματος. Ο δεύτερος τρόπος περιλαμβάνει την ανάγνωση και προβολή των επιπέδων που παρήγαγε το κάθε σύστημα και η αξιολόγηση τους με βάση την αισθητική και την ανθρώπινη κρίση για το εάν είναι ένα τεχνικά σωστό και πρωτότυπο επίπεδο.
\par
Χρησιμοποιήθηκαν και οι δύο τρόποι αξιολόγησης, μετρικές εφαρμόστηκαν και στο σύστημα του PCG και του ML αλλά παράλληλα αναπτύχθηκαν εργαλεία για την οπτικοποίηση των αποτελεσμάτων για να γίνεται ολοκληρωμένα και σφαιρικά η αξιολόγηση των επιπέδων που παράχθηκαν από το κάθε σύστημα. Και οι δύο τρόποι αξιολόγησης έχουν θετικά και αρνητικά στοιχεία, αλλά η συνδυαστική εφαρμογή τους φέρνει ένα καλύτερο αποτέλεσμα, ειδικά στον τομέα της δημιουργίας ηλεκτρονικών παιχνιδιών όπου η δημιουργικότητα και η ακρίβεια είναι το ίδιο σημαντικές μετρικές για το τελικό αποτέλεσμα.

\section{Αξιολόγηση συστήματος PCG}
\par
Το σύστημα PCG αναπτύχθηκε με βασικό στόχο την δημιουργία ενός dataset με εκατοντάδες δείγματα, συγκεκριμένης μορφής, αλλά διαφορετικών μεταξύ τους. Χρησιμοποιήθηκαν γνωστοί αλγόριθμοι με μερικές τροποποιήσεις και με την εισαγωγή μερικών μεταβλητών τυχαιότητας για την διαφοροποίηση των επιπέδων. Αναπτύχθηκε ένα σύστημα αξιολόγησης των επιπέδων με την απονομή βαθμών όταν τα παραγόμενα επίπεδα πληρούσαν κάποιες βασικές προϋποθέσεις που είχαν οριστεί. Αυτό το σύστημα αξιολόγησης ήταν ιδιαίτερα χρήσιμο όσο οι αλγόριθμοι του PCG αναπτύσσονταν, γιατί ήταν πολύ εύκολο να αξιολογηθεί εάν το αποτέλεσμα είναι το επιθυμητό ή όχι.
\par
Μόλις ολοκληρώθηκε η ανάπτυξη των αλγορίθμων του PCG και δεν υπήρχε συμπεριφορά που να μπορεί να δημιουργήσει κάτι πέρα από το επιθυμητό, το σύστημα αξιολόγησης ήταν πλέον αχρείαστο και απενεργοποιήθηκε για να μην καθυστερεί την εκτέλεση των αλγορίθμων. Μπορεί να ξανά χρησιμοποιηθεί εάν υπάρξει μελλοντική εξέλιξη της υλοποίησης.
\par
Το σύστημα PCG δημιουργεί με επιτυχία το είδος των επιπέδων που έχουμε ορίσει, τα αποθηκεύει με την επιθυμητή μορφή και τα διαβάζει για οπτικοποίηση όταν ζητηθεί, επιπλέον οπτικοποιεί τα επίπεδα που παράγονται από το ML για αξιολόγηση.



\section{Αξιολόγηση συστήματος ML}



\section{Αξιολόγηση συστήματος PCG και ML}



\section{Προτάσεις}