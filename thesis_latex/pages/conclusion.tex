\chapter{Συμπεράσματα}

\section{Περίληψη}
\par
Για την εξαγωγή των συμπερασμάτων μπορούμε να ξεχωρίσουμε δύο κύριες κατηγορίες προς αξιολόγηση. Η πρώτη κατηγορία αφορά το σύστημα του PCG και τις αποδόσεις του σε σχέση με τον στόχο για τον οποίο υλοποιήθηκε. Η δεύτερη κατηγορία αφορά τα δύο μοντέλα GAN που αναπτύχθηκαν και η μεταξύ τους σύγκριση με βάση τις μετρήσεις και τα αποτελέσματα που παρήγαγαν.
\par
Έχουμε δύο τρόπους αξιολόγησης των παραπάνω κατηγοριών. Ο πρώτος τρόπος περιλαμβάνει την καταγραφή των αποτελεσμάτων και την χρήση γνωστών μετρικών όπως είναι η ακρίβεια (accuracy) και το μέσο τετραγωνικό σφάλμα (mean square error) για την αξιολόγηση των GAN μοντέλων, καθώς και την χρήση κάποιας ορισμένης από εμάς μετρικής για την αξιολόγηση του PCG συστήματος. Ο δεύτερος τρόπος περιλαμβάνει την οπτικοποίηση των επιπέδων που δημιουργήθηκαν από το κάθε σύστημα και η αξιολόγηση τους με βάση την αισθητική και την ανθρώπινη κρίση για το εάν είναι ένα τεχνικά σωστό και πρωτότυπο επίπεδο.
\par
Χρησιμοποιήθηκαν και οι δύο τρόποι αξιολόγησης, μετρικές εφαρμόστηκαν στο σύστημα PCG και MLCG ενώ παράλληλα αναπτύχθηκαν εργαλεία για την οπτικοποίηση των αποτελεσμάτων για να γίνεται ολοκληρωμένα η αξιολόγηση των επιπέδων που δημιουργήθηκαν από το κάθε σύστημα. Και οι δύο τρόποι αξιολόγησης έχουν θετικά και αρνητικά στοιχεία, αλλά ο συνδυασμός και των δύο μεθόδων, καταλήγει σε ένα καλύτερο αποτέλεσμα, ειδικά στον τομέα της δημιουργίας ηλεκτρονικών παιχνιδιών όπου η δημιουργικότητα και η ακρίβεια είναι εξίσου σημαντικές μετρικές για το τελικό αποτέλεσμα.

\section{Αξιολόγηση συστήματος PCG}
\par
Το σύστημα PCG αναπτύχθηκε με βασικό στόχο την δημιουργία ενός dataset με εκατοντάδες δείγματα, συγκεκριμένης μορφής, αλλά και ταυτοχρόνως διαφορετικών μεταξύ τους. Χρησιμοποιήθηκαν γνωστοί αλγόριθμοι με μερικές τροποποιήσεις και με την εισαγωγή μεταβλητών τυχαιότητας για την διαφοροποίηση των επιπέδων. Επιπλέον αναπτύχθηκε ένα σύστημα αξιολόγησης των επιπέδων με την απονομή βαθμών όταν τα παραγόμενα επίπεδα πληρούσαν τις βασικές προϋποθέσεις που έχουν οριστεί. Αυτό το σύστημα αξιολόγησης ήταν ιδιαίτερα χρήσιμο όσο οι αλγόριθμοι του PCG αναπτύσσονταν, γιατί ήταν πολύ εύκολο να αξιολογηθεί εάν το αποτέλεσμα είναι το επιθυμητό ή όχι.
\par
Μόλις ολοκληρώθηκε η ανάπτυξη των αλγορίθμων του PCG και δεν υπήρχε συμπεριφορά που να μπορεί να δημιουργήσει κάτι πέρα από το επιθυμητό, το σύστημα αξιολόγησης ήταν πλέον αχρείαστο και απενεργοποιήθηκε για να μην καθυστερεί την εκτέλεση των αλγορίθμων. Μπορεί να ξανά χρησιμοποιηθεί εάν υπάρξει μελλοντική εξέλιξη της υλοποίησης.
\par
Το σύστημα PCG δημιουργεί με επιτυχία το είδος των επιπέδων που έχουμε ορίσει, τα αποθηκεύει με την επιθυμητή μορφή και τα διαβάζει για οπτικοποίηση όταν ζητηθεί, επιπλέον οπτικοποιεί τα επίπεδα που παράγονται από το ML για αξιολόγηση.



\section{Αξιολόγηση μοντέλων ML}
\par
Από τα δύο μοντέλα που αναπτύχθηκαν (Dense \& CNN), τα αποτελέσματα του CNN τόσο στις μετρήσεις όσο και οπτικά ήταν πολύ καλύτερα από τα αποτελέσματα του Dense μοντέλου. Τα καλύτερα αποτελέσματα που δημιουργήθηκαν ήταν πολύ παρόμοια με τα επίπεδα του dataset χωρίς όμως να έχουν κάποιο πρωτότυπο χαρακτηριστικό.
\par
Το CNN μοντέλο θα μπορούσε να χρησιμοποιηθεί ως γεννήτρια επιπέδων τέτοιου είδους με επιτυχία όπως δείχνουν τα δείγματα από τις τελικές εποχές εκπαίδευσης του μοντέλου.


\section{Προτάσεις}
\par
Πολλές ιδέες μπορούν να προταθούν για την μελλοντική ανάπτυξη και έρευνα αυτής της υλοποίησης. Μπορούμε να διακρίνουμε και πάλι τις προτάσεις σε δύο τμήματα  αντίστοιχα με αυτά της αξιολόγησης.
\par
Το σύστημα του PCG μπορεί να αναπτυχθεί για να περιέχει παραπάνω λεπτομέρειες και λειτουργίες στα επίπεδα που παράγει ώστε να μοιάζουν πιο ολοκληρωμένα. Για παράδειγμα η προσθήκη εισόδων στο επίπεδο και στα δωμάτια που περιέχει είναι μια πολύ ενδιαφέρον προγραμματιστική πρόκληση και απαραίτητη λειτουργία για να προστεθεί το επίπεδο σε κάποιο ηλεκτρονικό παιχνίδι. Επιπλέον μπορεί να γίνει η προσθήκη αντικειμένων για να βελτιωθεί η αισθητική και η διαφορετικότητα των επιπέδων. Τέλος μπορεί να γίνει και ο προγραμματισμός αποστολών μέσα στο επίπεδο όπου αντικείμενα πρέπει να τοποθετηθούν με κάποια ιεραρχία στον χώρο.
\par
Το σύστημα του MLCG μπορεί να ακολουθεί την εξέλιξη του συστήματος του PCG, το οποίο σημαίνει ότι θα πρέπει να αλλάξει η αρχιτεκτονική του ώστε να προσαρμοστεί στην αυξημένη πολυπλοκότητα των νέων επιπέδων. Παράλληλα όμως, μπορεί να αναπτυχθεί και με την εξερεύνηση και δοκιμή νέων αρχιτεκτονικών, που μπορεί να αποδειχθούν καλύτερα από την υπάρχον υλοποίηση.
\par
Ένας μεγάλος στόχος που δεν επιτεύχθηκε σε αυτή την υλοποίηση είναι η δημιουργία ενός συστήματος MLCG που θα "ξεφύγει" από τα χαρακτηριστικά του dataset και θα καταφέρει να δημιουργήσει ένα πρωτότυπο ΚΑΙ λειτουργικό επίπεδο. Αυτή παραμένει ακόμα μια μεγάλη πρόκληση στον τομέα του Content Generation.









