\chapter{Εισαγωγή}

\par
Το αντικείμενο της δημιουργίας περιεχομένου(Content Generation) είναι ένα πεδίο της Τεχνητής Νοημοσύνης με εφαρμογές στην ανάπτυξη ηλεκτρονικών παιχνιδιών, στην μουσική, στις ταινίες και σε πολλούς ακόμα τομείς. Μπορούμε να διακρίνουμε δύο διαφορετικές κατηγορίες δημιουργίας περιεχομένου με βάση τα είδη των αλγορίθμων που εφαρμόζονται σε κάθε κατηγορία. Αυτή η εργασία ερευνά τις δύο αυτές κατηγορίες και την αλληλεπίδραση που μπορεί να δημιουργηθεί μεταξύ τους.
\par
Η πρώτη κατηγορία αφορά τη δημιουργία περιεχομένου με τη χρήση ντετερμινιστικών αλγορίθμων (Procedural Content Generation), και η δεύτερη κατηγορία περιέχει την δημιουργία περιεχομένου με την χρήση μηχανικής μάθησης (Machine Learning). Η μέθοδος του Procedural Content Generation (PCG), είναι η παλαιότερη από τις δύο και αυτή με τις περισσότερες εφαρμογές και ερευνητικό περιεχόμενο. Η μέθοδος του Machine Learning (ML) έχει γνωρίσει μεγάλη ανάπτυξη τα τελευταία χρόνια και αποδίδει εντυπωσιακά αποτελέσματα όπου εφαρμόζεται με σωστή έρευνα και ανάπτυξη.

\section{Στόχοι της εργασίας}
\par
Ο βασικός στόχος αυτής της εργασίας είναι η ανάπτυξη δύο συστημάτων παραγωγής δισδιάστατων επιπέδων παιχνιδιού με μια συγκεκριμένη διάταξη. Εφαρμόστηκαν αλγόριθμοι και από τις δύο κατηγορίες για την δημιουργία ενός απλού δισδιάστατου (2D) επιπέδου. Το επίπεδο έχει πολλά κοινά με πραγματικά επίπεδα όπως αυτά αναπαρίστανται σε 2D ηλεκτρονικά παιχνίδια. Περιέχει δωμάτια, διαδρόμους και τοίχους, με μια συγκεκριμένη διάταξη στον διαθέσιμο χώρο του επιπέδου.
\par
Αρχικός στόχος είναι η ανάπτυξη ενός PCG συστήματος παραγωγής επιπέδων με την διάταξη που έχουμε ορίσει. Τα επίπεδα πρέπει να είναι διαφορετικά το ένα από το άλλο όσο περισσότερο γίνεται χωρίς όμως να έχουν διαφορετική διάταξη. Το σύστημα πρέπει να μπορεί να παράγει και να αποθηκεύει με μια συγκεκριμένη μορφή, πολλά επίπεδα σε μικρό χρονικό διάστημα. Το PCG σύστημα είναι ο δημιουργός του συνόλου δεδομένων εκπαίδευσης που θα χρησιμοποιηθεί στο ML σύστημα.
\par
Τελικός στόχος της εργασίας είναι η ανάπτυξη ενός ML συστήματος, και συγκεκριμένα ενός Generative Adversarial Network (GAN), συστήματος, το οποίο θα μπορεί να διαβάζει τα επίπεδα που δημιουργήθηκαν από τον PCG σύστημα και να εκπαιδεύεται από αυτά ώστε να μπορεί να παράγει αντίστοιχα μόλις του ζητηθεί.
\par
Αν και η αναπαράσταση των επιπέδων κρατήθηκε πολύ απλή, υπάρχουν πολλές δυσκολίες που πρέπει να επιλυθούν και για τις δύο κατηγορίες ώστε να παραχθούν ικανοποιητικά αποτελέσματα.

\section{Προκλήσεις και δυσκολίες}
\par
Η μεγαλύτερη πρόκληση που αντιμετωπίστηκε κατά την ανάπτυξη αυτής της εργασίας ήταν η σχεδίαση της αρχιτεκτονικής που θα έχει το μοντέλο του GAN στο ML σύστημα. Το GAN είναι ένα είδος νευρωνικού δικτύου, και η αρχιτεκτονική του αφορά το είδος, την ιεραρχία και τις παραμέτρους των επιπέδων που το αποτελούν. Η αρχιτεκτονική του νευρωνικού δικτύου αλλάζει ραγδαία ανάλογα με το είδος της εφαρμογής στην οποία θα εφαρμοστεί. Για αυτού του είδος τις εφαρμογές, \textit{ανάπτυξη περιεχομένου}, δεν υπάρχουν πολλές υλοποιήσεις διαθέσιμες, ειδικά με τα κριτήρια που απαιτούνται στο συγκεκριμένο πρόβλημα. Συνεπώς η αρχιτεκτονική του δικτύου βρέθηκε μετά από πολλές δοκιμές και πειραματισμούς.
\par
Όπως αναφέρθηκε, η κατηγορία της δημιουργία περιεχομένου με την χρήση μηχανικής μάθησης είναι πολύ πιο πρόσφατη και λιγότερο διαδεδομένη από την κατηγορία του PCG. Συνεπώς έχει λιγότερα παραδείγματα και βιβλιογραφία από την οποία μπορούμε να πάρουμε ιδέες και λύσεις. Αυτή η δυσκολία είναι που την κάνει και πιο ενδιαφέρουσα.
\par
Μια ακόμα δυσκολία που αντιμετωπίστηκε ήταν η μεταφορά των επιπέδων μεταξύ συστημάτων. Τα δύο συστήματα, PCG και ML, είναι ανεπτυγμένα σε διαφορετικές γλώσσες προγραμματισμού το καθένα, έχουν διαφορετικά είδη δεδομένων και έπρεπε να μπορούν να διαβάζουν και να αποθηκεύουν τα επίπεδα με μια κοινή μορφή. Αυτό είχε μια επιπλέον δυσκολία για το ML σύστημα καθώς τα δεδομένα πρέπει να περάσουν από ένα στάδιο προ επεξεργασίας, το οποίο τα αλλοιώνει, συνεπώς πριν την αποθήκευση τους από το ML σύστημα έπρεπε να περάσουν από την αντίστοιχη προ επεξεργασία με την αντίθετη μετατροπή για να είναι αναγνώσιμα από το σύστημα PCG.


\section{Περιεχόμενα}
Στο κεφάλαιο 2 γίνεται μια περιγραφή της θεωρίας και εννοιολογίας του PCG καθώς και μερικών από των πιο αντιπροσωπευτικών αλγορίθμων. Στο κεφάλαιο 3 αναλύεται πως υλοποιήθηκε το σύστημα PCG σε αυτή την εργασία, σε αναφορά και με το προηγούμενο κεφάλαιο. Με την ίδια λογική, στο κεφάλαιο 4 γίνετε μια θεωρητική ανάλυση του τομέα του ML, με έμφαση στις μεθόδους που χρησιμοποιήθηκαν στην υλοποίηση η οποία αναλύεται στο κεφάλαιο 5 μαζί με τα αποτελέσματα που παρήχθησαν από την εκπαίδευση του συστήματος ML. Στο προτελευταίο κεφάλαιο, 6, αναλύουμε τα συμπεράσματα που προέκυψαν από αυτή την εργασία και μελλοντικές προτάσεις. Το τελευταίο κεφάλαιο, 7, περιέχει αναφορές στην βιβλιογραφία και συνδέσμους σε πηγές για την θεωρητική και τεχνική τεκμηρίωση της εργασίας.




