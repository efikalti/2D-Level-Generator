% print no page number
\thispagestyle{empty}

\chapter{Service layer - Restful API}

\par
Τα κομμάτια του frontend και του backend, παρόλο που είναι ανεξάρτητα το ένα από το άλλο, για την σωστή λειτουργία της εφαρμογής είναι απαραίτητο να επικοινωνήσουν μεταξύ τους. Την επικοινωνία τους αναλαμβάνει ένα τρίτο κομμάτι της εφαρμογής που ονομάζετε \textbf{Service Layer}.\par Σε πολλές εφαρμογές, το \textbf{Service Layer} υλοποιήτε από το κομμάτι του backend καθώς οι ίδιες μέθοδοι που κάνουν την επικοινωνία backend με frontend κάνουν και την επικοινωνία backend με τη βάση δεδομένων.Και σε αυτήν την περίπτωση το \textbf{Service Layer} έχει υλοποιηθεί μαζί με το backend.\par Σε γενικές γραμμές, πρόκειται για ένα σύνολο σαφώς καθορισμένων μεθόδων επικοινωνίας το οποίο για συντομία το λέμε \textbf{API (application programming interface)}.Το API ορίζει τον σωστό τρόπο για έναν προγραμματιστή να γράψει ένα πρόγραμμα που ζητά υπηρεσίες από λειτουργικό σύστημα (OS) ή άλλη εφαρμογή. Τα API υλοποιούνται με λειτουργικές κλήσεις που αποτελούνται από ρήματα και ουσιαστικά. Η απαιτούμενη σύνταξη περιγράφεται συνήθως στο documentation της κάθε εφαρμογής.


% leave 60mm empty space below
\vspace{60mm}

\section{Περιγραφή}
	Τα API αποτελούνται από δύο στοιχεία. Το πρώτο είναι ένα σύνολο κανόνων που περιγράφει τον τρόπο ανταλλαγής πληροφοριών μεταξύ προγραμμάτων, που γίνεται με τη μορφή αίτησης για επεξεργασία και επιστροφής των απαραίτητων δεδομένων. Το δεύτερο είναι το λογισμικού γραμμένο με τις προδιαγραφές που ορίστηκαν από τους κανόνες και δημοσιεύεται για χρήση.

Το λογισμικό που επιθυμεί να αποκτήσει πρόσβαση στις δυνατότητες και τις δυνατότητες του API λέγεται ότι καλεί το API και το λογισμικό που υλοποιεί το API λέγεται ότι το δημοσιεύει.

\section{Σχεδίαση του API}
	Ο σωστός σχεδιασμός του API βασίζετε στις αρχές της διαφάνειας και της ασφάλειας των δεδομένων. Το API θα πρέπει να αποκρύπτει τις μεθόδους και την πολυπλοκότητα της επικοινωνίας από και προς τις δύο υπηρεσίες.Έτσι, ο σχεδιασμός του API επιχειρεί να παρέχει μόνο τα εργαλεία που θα περίμενε να παραλάβει το frontend.
\newline

\section{Είδη APIs}
Υπάρχουν τριών ειδών APIs. 

\begin{description}
\item[Local APIs] Eίναι η αρχική μορφή, από την οποία ήρθε το όνομα. Προσφέρουν υπηρεσίες OS ή middleware σε προγράμματα εφαρμογών. Το API της Microsoft .NET, το TAPI (API τηλεφωνίας) για φωνητικές εφαρμογές και τα API πρόσβασης σε βάσεις δεδομένων είναι  Local API. 
\item[Web APIs] Αυτά τα API έχουν σχεδιαστεί για να αντιπροσωπεύουν ευρέως χρησιμοποιούμενους πόρους όπως οι σελίδες HTML και έχουν πρόσβαση μέσω ενός απλού πρωτοκόλλου HTTP. Οποιαδήποτε διαδικτυακή διεύθυνση URL καλεί ένα web API. Τα API του Web ονομάζονται συχνά REST  ή RESTful επειδή ο εκδότης των διασυνδέσεων REST δεν αποθηκεύει εσωτερικά δεδομένα μεταξύ αιτήσεων. Ως εκ τούτου, τα αιτήματα πολλών χρηστών μπορούν να αναμειχθούν όπως και στο διαδίκτυο.
\item[Program APIs] Τα API προγραμμάτων βασίζονται σε τεχνολογία κλήσης εξ αποστάσεως (Remote Procedure Call) (RPC), που καθιστά ένα εξ αποστάσεως συστατικό του προγράμματος να φαίνεται τοπικό για το υπόλοιπο λογισμικό. Για παράδειγμα η σειρά API της Microsoft WS.
\end{description}

Καθώς η συγκεκριμένη εφαρμογή έχει επικοινωνία με web application έχει υλοποιηθεί ένα web API.

\section{REST - RESTful API}
Οι REST (Representational state transfer) ή RESTful υπηρεσίες, όπως η συγκεκριμένη, αποτελούν έναν τρόπο παροχής λειτουργικότητας μεταξύ συστημάτων στο Διαδίκτυο. Οι υπηρεσίες Web που είναι συμβατές με τους κανόνες που ορίζει το REST επιτρέπουν στα συστήματα να έχουν πρόσβαση και να χειρίζονται τα δεδομένα του διαδικτύου χρησιμοποιώντας ένα ομοιόμορφο και προκαθορισμένο σύνολο λειτουργιών. Υπάρχουν και άλλες μορφές υπηρεσιών με τους δικούς τους κανόνες και λειτουργίες, όπως το WSDL και το SOAP.
\par Σε μια RESTful υπηρεσία , τα αιτήματα \textit{(requests)} που υποβάλλονται στο URL ενός πόρου θα προκαλέσουν μια απάντηση \textit{(response)} που μπορεί να είναι XML, HTML, JSON ή κάποια άλλη καθορισμένη μορφή. Η απόκριση μπορεί να επιβεβαιώσει ότι έχουν γίνει ορισμένες αλλαγές στον αποθηκευμένο πόρο και μπορεί να παρέχει συνδέσεις υπερκειμένου με άλλους σχετικούς πόρους ή συλλογές πόρων. Χρησιμοποιώντας HTTP, όπως είναι πιο συνηθισμένο, το είδος των διαθέσιμων λειτουργιών περιλαμβάνει εκείνες που είναι προκαθορισμένες από τα ρήματα HTTP GET, POST, PUT, DELETE κ.τ.λ

\subsection{Χαρακτηριστικά του REST}
Οι κανόνες που ορίζονται από το REST εξυπηρετούν στην διασφάλιση ορισμένων πλεονεκτημάτων που ορίζουν τις RESTful υπηρεσίες όταν υλοποιούνται σωστά.Τα παρακάτω είναι μερικά από τα χαρακτηριστικά που προσπαθεί να διασφαλίσει το REST
\begin{itemize}
\item Performance 
\item Scalability
\item Simplicity
\item Modifiability
\item Reliability
\end{itemize}


\subsection{Από URL σε HTTP}
Ένας από τους πιο βασικούς κανόνες που ορίζει το REST είναι η αντιστοίχηση των URL με συγκεκριμένα HTTP ρήματα και λειτουργίες.
Ο παρακάτω πίνακας περιγράφει ένα από τα αντικείμενα της εφαρμογής, το \textit{job} και πως αντιστοιχίζονται τα URL της web εφαρμογής με τα HTTP calls του API.

\begin{center}
\begin{tabular}{ |c|c|c|c| } 
\hline
URL & HTTP GET & col3 \\
\hline
http://{backend}/api/jobs/ & cell5 & cell6 \\ 
2 & cell8 & cell9 \\ 
\hline
\end{tabular}
\end{center}







