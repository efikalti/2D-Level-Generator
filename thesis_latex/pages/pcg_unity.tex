% print no page number
\thispagestyle{empty}

\chapter{Procedural Content Generation with Unity}

Την υλοποίηση αυτής της εργασίας μπορούμε να την χωρίσουμε σε δύο ξεχωριστές υλοποιήσεις. Την υλοποίηση ενός \textit{Procedural Content Generation} συστήματος όπως αυτά που περιγράψαμε στο προηγούμενο κεφάλαιο, και σε ένα \textit{Machine Learning Content Generation} σύστημα. Σε αυτό το κεφάλαιο αναλύουμε την τεχνική υλοποίηση του PCG.
\par
Μέχρι τώρα έχουμε αναλύσει την θεωρία πίσω από το αντικείμενο του PCG, καινούργιες προκλήσεις και προβλήματα προκύπτουν όταν προσπαθούμε να περάσουμε από την θεωρία στην πράξη, κάτι που συμβαίνει σε όλους τους επιστημονικούς τομείς και το PCG δεν αποτελεί  εξαίρεση.
\par
Από το στάδιο της αρχικής ιδέας μέχρι την έναρξη της υλοποίησης μπορούμε να διακρίνουμε ένα ενδιάμεσο στάδιο, του σχεδιασμού και της έρευνας σχετικά με τις τεχνολογίες, τις δομές δεδομένων και τους αλγορίθμους που θα κληθούμε να υλοποιήσουμε. Αυτή η προσέγγιση ισχύει για όλες τις υλοποιήσεις της επιστήμης της πληροφορικής και ειδικότερα του game development όπου έχουμε πολλά υποσυστήματα που θα πρέπει να λειτουργήσουν παράλληλα και να συνεργαστούν με το σύστημα του PCG. Οπότε η σωστή οργάνωση και σχεδιασμός της αρχιτεκτονικής του νωρίς είναι πολύ κρίσιμο κομμάτι.
\par
Κατά την υλοποίηση και ιδιαίτερα κατά το στάδιο της αξιολόγησης όπως είδαμε μπορεί να προκύψουν προβλήματα όπως μη αποδεκτά αποτελέσματα, γεγονός που μπορεί να οδηγήσει στον ανασχεδιασμό και την επιλογή διαφορετικών αλγορίθμων ή παραμέτρων. Αυτή η διαδικασία μπορεί να επαναληφθεί πολλές φορές καθώς, όπως ειπώθηκε και προηγουμένως, τα \textit{καλά} συστήματα PCG βγαίνουν μέσα από δοκιμές και λάθη (\textit{trial and error}).
\par
Παρόλαυτα, η υλοποίηση ενός τέτοιου συστήματος μπορεί να θεωρηθεί, και είναι και η προσωπική εμπειρία της συγγραφέας, ως μια πολύ δημιουργική εργασία, ειδικά σε σύγκριση με την διαδικασία ανάπτυξης άλλων ειδών λογισμικού. Πρόβληματα που απαιτούν σκέψη \textit{outside of the box} και δημιουργηκότητα, σε συνδυασμό με πολύ καλή γνώση της θεωρίας και της τεχνολογίας είναι ένα από τα χαρακτηριστικά που κάνουν αυτή την ανάπτυξη λογισμικού τόσο ενδιαφέρουσα και ιδιαίτερη.
Είδαμε μερικούς από τους αλγόριθμους που μπορούν να χρησιμοποιηθούν για την δημιουργία περιεχομένου. Υπάρχουν πολλοί περισσότεροι, ο καθένας με τις παραλλαγές του. Συνεπώς είναι πολύ εύκολο να δημιουργήσουμε περιεχόμενο, το δύσκολο κομμάτι του PCG είναι η δημιουργία \textbf{καλού} περιεχομένου. Σε αυτή την ενότητα θα αναλύσουμε πως γίνετε την αξιολόγηση PCG συστημάτων με τη χρήση άλλων συστημάτων, των evaluators.

\section{Game Engines}
Οι Μηχανές Δημιουργίας Ηλεκτρονικών Παιχνιδιών (\textit{Game Engines}) είναι περιβάλλοντα υλοποίησης παιχνιδιών, όπως περιγράφει και το όνομα τους. Είναι πολύπλοκα υπολογιστικά συστήματα που χρησιμοποιούνται από εταιρείες και ομάδες ανθρώπων σε όλο τον κόσμο για την υλοποίηση όλων των ειδών τα ηλεκτρονικά παιχνίδια. Συνήθως περιλαμβάνουν συστήματα για να βοηθήσουν τους προγραμματιστές και σχεδιαστές στην υλοποίηση, όπως το σύστημα προσομοίωσης φυσικών καταστάσεων και νόμων (ταχύτητα, επιτάχυσνη, βαρύτητα).
\par
Άλλα τέτοια συστήματα που έχουν τα περισσότερα Game Engines είναι συστήματα για την καταγραφή και επεξεργασία των εισόδων που δίνει ο χρήστης (πάτημα κουμπιού, κίνηση ποντικιού κ.ά), συστήματα για την εμφάνιση γραφικών (\textit{textures}, \textit{sprites}), συστήματα για την κίνηση αντικειμένων με γραφικά (\textit{animations}) και πολλά άλλα.
\par
Υπάρχουν Game Engines που εξειδικεύονται στην δημιουργία ενός συγκεκριμένου είδους παιχνιδιού, όπως για παράδειγμα το \textit{Hero Engine} εξειδικεύεται στην δημιουργία παιχνιδιών που παίξονται μέσω Internet (\textit{Online video games}). Αυτό σημαίνει ότι το Hero Engine έχει πολύ καλά ανεπτυγμένα συστήματα για την επικοινωνία \textit{client-server} εφαρμογών σε πραγματικό χρόνο καθώς και την διαχείριση δεδομένων που αποθηκεύονται κεντρικά σε κάποιο \textit{server}.
\par
Επίσης υπάρχουν και τα Game Engines που δεν φαίνετε να έχουν κάποια εξειδίκευση σε κανένα είδος, αλλά υποστηρίζουν ότι είδους game και αν θέλουμε να ανατπύξουμε. Αυτά τα Game Engine έχουν γνωρίσει μεγάλη δημοτικότητα τα τελευταία χρόνια, σε συνδυασμό με το γεγονός ότι είναι διαθέσιμα για όλους, με αποτέλεσμα να διευρυνθεί η χρήση τους με μεγάλη επιτυχία από ομάδες όλων των μεγεθών (μικρά independent studio και developers μέχρι μεγάλες εταιρείες). Τέτοια Game Engines είναι το Unity Engine, το Unreal Engine, το Godot Engine (το οποίο είναι OpenSource) κ.ά.
\par
Η εκμάθηση ενός τέτοιου προγράμματος παίρνει χρόνο και μελέτη, καθώς το κάθε Game Engine έχει δικές του υλοποιήσεις για το κάθε σύστημα, διαφορετικό περιβάλλον αλληλεπίδρασης με τον χρήστη (\textit{editor}) καθώς και διαφορετική γλώσσα προγραμματισμού που πρέπει να χρησιμοποιήσει ο προγραμματιστής. Υπάρχουν πολλά κριτήρια για την επιλογή ποιού Game Engine θα χρησιμοποιηθεί για την υλοποίηση ενός παιχνιδιού ή συστήματος, όπως στη δική μας περίπτωση. Τα πιο σημαντικά είναι:
\begin{itemize}
  \item Οι γνώσεις της ομάδας (Game Engine, γλώσσα προγραμματισμού)
  \item Οι
\end{itemize}

\begin{description}
\item[$\bullet$ Οι γνώσεις της ομάδας] Εάν έχουν προηγούμενη εμπειρία με κάποιο από τα διαθέσιμα Game Engines και με τις υποστηριζόμενες γλώσσες προγραμματισμού
\item[$\bullet$ Οι δυνατότητες του Game Engine] Εάν ο στόχος είναι η υλοποίησης μιας εφαρμογής για κινητές συσκευές, πρέπει να επιλεχθεί ένα Game Engine που να υποστηρίζει αυτή την πλατφόρμα.
\item[$\bullet$ Το κόστος του Game Engine] Πολλά Game Engines δεν παρέχονται ελεύθερα ή για να έχουμε πρόσβαση σε κάποιες λειτουργίες τους χρειάζετε η καταβολή κάποιου ποσού.
\end{description}

\par
Σε αυτή την υλοποίηση επιλέχθηκε το \textbf{Unity Game Engine} καθώς η συγγραφέας το γνωρίζει όπως και την γλώσσα προγραμματισμού \textbf{C\#} που χρησιμοποιήθηκε. Επιπλέον παρέχετε δωρεάν υπό προυποθέσεις.