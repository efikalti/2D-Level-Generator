% print no page number
\thispagestyle{empty}

\chapter{Machine Learning}

Η Μηχανική Μάθηση είναι ένας τομέας της Τεχνητής Νοημοσύνης που έχει ως στόχο την εκπαίδευση συστημάτων για την λήψη αποφάσεων, εκτέλεση ενεργειών και δημιυργία δεδομένων χωρίς την παρέμβαση κάποιου ανθρώπινου παράγοντα. Στο κέντρο της θεωρίας της Μηχανικής Μάθησης βρίσκονται τα μοντέλα (\textit{models}) τα οποία είναι μαθηματικές αναπαραστάσεις που δημιουργούνται από τα δεδομένα (\textit{training sample}) του συστήματος. Με βάση αυτά τα μοντέλα, το σύστημα καλείται να "πράξει" σε καινούργια και άγνωστα δεδομένα..
\par
Ως επιστήμη χρησιμοποιεί μαθηματικά μοντέλα και στατιστική για να φτιάξει μια αναπαράσταση των δεδομένων, να βρει \textit{patterns} και να τα κατηγοριοποιήσει με στόχο να εκτελέσει ενέργειες χωρίς να χρειαστεί οδηγίες από κάποιον άνθρωπο ή άλλο σύστημα.
\par
Ως τομέας, η Μηχανική Μάθηση δημιουργήθηκε από την έρευνα σε στόχο την δημιουργία τεχνητής νοημοσύνης στα υπολογιστικά συστήματα. Η ιδέα πίσω από την Μηχανική Μάθηση είναι ιδιαίτερα απλή, η εκμάθηση μέσα από δεδομένα, περιγράφει πολύ γενικά το πως μαθαίνουμε και εμείς οι άνθρωποι αλλά και όλοι οι έμβιοι οργανισμοί. Στην υλοποίηση της είναι ένα πολύπλοκο πρόβλημα με χρόνια μελέτης και έρευνας που ακόμα παραμένει άλυτο στο μεγαλύτερο του βαθμό.
\par
Έχει γνωρίσει τεράστια ανάπτυξη και υιοθέτηση τα τελευταία χρόνια χάρη στην ραγδαία ανάπτυξη του \textit{hardware} των υπολογιστών το οποίο οδήγησε σε μεγάλα βήματα τον τομέα των νευρωνικών δικτύων και συγκεκριμένα των \textit{Deep Neural Networks}. Έχει ξεκινήσει να εφαρμόζετε σε κάθε τομέα του σύχρονου κόσμου, όπως στην ιατρική, την φαρμακευτική, την οικονομία και την διοίκηση με αποτελέσματα που ήταν ανέφικτα πριν μερικά χρόνια


\section{Machine Learning \& Games}
Η Μηχανική Μάθηση και τα παιχνίδια έχουν μια στενή σχέση από την στιγμή της διμουργίας του πεδίου μέχρι και σήμερα. Μερικές από τις πρώτες έρευνεσ και εφαρμογές της ήταν στην εκμάθηση παιχνιδιών όπως το σκάκι και η ντάμα, με στόχο να είναι σε θέση να ανταγωνιστεί ανθρώπινους αντιπάλους, χωρίς την ανάγκη να προγραμματιστούν κανόνες και κινήσεις για το παιχνίδι. 
\par
Ένα από τα επιτεύγματα που έδωσαν πολύ δημοσιότητα στα \textit{Deep Neural Networks} είναι η δημιουργία του συστήματος \textit{AlphaGo} ένα \textit{Deep Neural Network} το οποίο κατάφερε να νικήσει πρωταθλητές του επιτραπέζιου παιχνιδιού \textit{Gο} χωρίς \textit{handicap} σε ταμπλό 19x19. Ως επίτευγμα είναι ιδιαίτερα εντυπωσιακό καθώς ο αριθμός κινήσεων και στρατηγικών αυτού του παιχνιδιού ήταν, και παραμένει, υπολογιστικά ασύλλιπτα μεγάλος για να προγραμματιστεί πρόγραμμα που να παίζει αποτελεσματικά. Συνεπώς η επίτευξη του με την χρήση \textit{Deep Neural Networks} ήταν μεγάλο σημείο για την επιστημονική κοινώτητα για να αναγνωρίσει και να αρχίσει να εφαρμόζει Μηχανική Μάθηση και σε άλλα, υπολογιστικά άλυτα προβλήματα.
\par
Εκτός από την εκπαίδευση ενός συστήματος για να παίζει ένα παιχνίδι, η Μηχανική Μάθηση μπορεί να εφαρμοστεί, όπως και έγινε σε αυτή την εργασία, για την δημιουργία κομματιών ενός παιχνιδιού. Υπάρχουν υλοποιησεις που βασίζονται σε διάφορους αλγορίθμους της Μηχανικής Μάθησης για την δημιουργία περιεχομένου όπως το έχουμε περιγράψει παραπάνω.
\par
Επιπλέον μεγάλη εξάπλωση έχει παρατηρηθεί και στην χρήση Μηχανικής Μάθησης για το \textit{player modeling}. Την κατηγοριοποίηση δηλαδή του παίκτη με βάση τον τρόπο που παίζει για την εξαγωγή συμπερασμάτων, αν ο παίκτης παίζει τίμια για παράδειγμα, και την δημιουργία περιεχομένου που ταιριάζει σε αυτόν τον παίκτη, με σκοπό την μεγιστοποίηση της διασκέδασης και απόλαυσης που του παρέχει το παιχνίδι.


\section{Machine Learning στα Video Games}
Αν και πολλοί τομείς της Τεχνητής Νοημοσύνης έχουν χρησιμοποιηθεί και συνεχίζουν να χρησιμοποιούνται με μεγάλη επιτυχία στην βιομηχανία των \textit{Video Games}, η Μηχανική Μάθηση δεν έχει την χρήση και δημοτικότητα που θα περιμέναμε απότι είναι γνωστό. Υπάρχουν πολλές θεωρίες σχετικά με αυτό το θέμα όπως θα δούμε. 
\par
Λόγω της μεγάλης ανταγωνιστικότητας και για την προστασία της τεχνολογικής γνώσης, οι εταιρείες του \textit{Game Development} δεν συνηθίζουν να περιγράφουν ή να ανακοινώνουν τα συστήματα τους και με τι αλγορίθμους υλοποιήθηκαν. Συνεπώς είναι πολύ δύσκολο να γνωρίζουμε με σιγουριά αν και που χρησιμοποιείται Μηχανική Μάθηση. Τα παιχνίδια που έχουν ή χρησιμοποιούν Μηχανική Μάθηση είναι ένα πολύ μικρό υποσύνολο των παιχνιδιών που υπάρχουν , γιαυτό υπάρχει η θεωρία ότι ο τομέας της Μηχανικής Μάθησης δεν εφαρμόζετε, όπως θα περιμέναμε από τον κόσμο του \textit{Game Development}









