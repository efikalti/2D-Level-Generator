% print no page number
\thispagestyle{empty}

\chapter{Machine Learning}

Η Μηχανική Μάθηση είναι ένας τομέας της Τεχνητής Νοημοσύνης που έχει ως στόχο την εκπαίδευση συστημάτων για την λήψη αποφάσεων, εκτέλεση ενεργειών και δημιυργία δεδομένων χωρίς την παρέμβαση κάποιου ανθρώπινου παράγοντα. Στο κέντρο της θεωρίας της Μηχανικής Μάθησης βρίσκονται τα μοντέλα (\textit{models}) τα οποία είναι μαθηματικές αναπαραστάσεις που δημιουργούνται από τα δεδομένα (\textit{training sample}) του συστήματος. Με βάση αυτά τα μοντέλα, το σύστημα καλείται να "πράξει" σε καινούργια και άγνωστα δεδομένα..
\par
Ως επιστήμη χρησιμοποιεί μαθηματικά μοντέλα και στατιστική για να φτιάξει μια αναπαράσταση των δεδομένων, να βρει \textit{patterns} και να τα κατηγοριοποιήσει με στόχο να εκτελέσει ενέργειες χωρίς να χρειαστεί οδηγίες από κάποιον άνθρωπο ή άλλο σύστημα.
\par
Ως τομέας, η Μηχανική Μάθηση δημιουργήθηκε από την έρευνα σε στόχο την δημιουργία τεχνητής νοημοσύνης στα υπολογιστικά συστήματα. Η ιδέα πίσω από την Μηχανική Μάθηση είναι ιδιαίτερα απλή, η εκμάθηση μέσα από δεδομένα, περιγράφει πολύ γενικά το πως μαθαίνουμε και εμείς οι άνθρωποι αλλά και όλοι οι έμβιοι οργανισμοί. Στην υλοποίηση της είναι ένα πολύπλοκο πρόβλημα με χρόνια μελέτης και έρευνας που ακόμα παραμένει άλυτο στο μεγαλύτερο του βαθμό.
\par
Έχει γνωρίσει τεράστια ανάπτυξη και υιοθέτηση τα τελευταία χρόνια χάρη στην ραγδαία ανάπτυξη του \textit{hardware} των υπολογιστών το οποίο οδήγησε σε μεγάλα βήματα τον τομέα των νευρωνικών δικτύων και συγκεκριμένα των \textit{Deep Neural Networks}. Έχει ξεκινήσει να εφαρμόζετε σε κάθε τομέα του σύχρονου κόσμου, όπως στην ιατρική, την φαρμακευτική, την οικονομία και την διοίκηση με αποτελέσματα που ήταν ανέφικτα πριν μερικά χρόνια


\section{Machine Learning \& Games}
Η Μηχανική Μάθηση και τα παιχνίδια έχουν μια στενή σχέση από την στιγμή της διμουργίας του πεδίου μέχρι και σήμερα. Μερικές από τις πρώτες έρευνεσ και εφαρμογές της ήταν στην εκμάθηση παιχνιδιών όπως το σκάκι και η ντάμα, με στόχο να είναι σε θέση να ανταγωνιστεί ανθρώπινους αντιπάλους, χωρίς την ανάγκη να προγραμματιστούν κανόνες και κινήσεις για το παιχνίδι. 
\par
Ένα από τα επιτεύγματα που έδωσαν πολύ δημοσιότητα στα \textit{Deep Neural Networks} είναι η δημιουργία του συστήματος \textit{AlphaGo} ένα \textit{Deep Neural Network} το οποίο κατάφερε να νικήσει πρωταθλητές του επιτραπέζιου παιχνιδιού \textit{Gο} χωρίς \textit{handicap} σε ταμπλό 19x19. Ως επίτευγμα είναι ιδιαίτερα εντυπωσιακό καθώς ο αριθμός κινήσεων και στρατηγικών αυτού του παιχνιδιού ήταν, και παραμένει, υπολογιστικά ασύλλιπτα μεγάλος για να προγραμματιστεί πρόγραμμα που να παίζει αποτελεσματικά. Συνεπώς η επίτευξη του με την χρήση \textit{Deep Neural Networks} ήταν μεγάλο σημείο για την επιστημονική κοινώτητα για να αναγνωρίσει και να αρχίσει να εφαρμόζει Μηχανική Μάθηση και σε άλλα, υπολογιστικά άλυτα προβλήματα.
\par
Εκτός από την εκπαίδευση ενός συστήματος για να παίζει ένα παιχνίδι, η Μηχανική Μάθηση μπορεί να εφαρμοστεί, όπως και έγινε σε αυτή την εργασία, για την δημιουργία κομματιών ενός παιχνιδιού. Υπάρχουν υλοποιησεις που βασίζονται σε διάφορους αλγορίθμους της Μηχανικής Μάθησης για την δημιουργία περιεχομένου όπως το έχουμε περιγράψει παραπάνω.
\par
Επιπλέον μεγάλη εξάπλωση έχει παρατηρηθεί και στην χρήση Μηχανικής Μάθησης για το \textit{player modeling}. Την κατηγοριοποίηση δηλαδή του παίκτη με βάση τον τρόπο που παίζει για την εξαγωγή συμπερασμάτων, αν ο παίκτης παίζει τίμια για παράδειγμα, και την δημιουργία περιεχομένου που ταιριάζει σε αυτόν τον παίκτη, με σκοπό την μεγιστοποίηση της διασκέδασης και απόλαυσης που του παρέχει το παιχνίδι.


\section{Machine Learning στα Video Games}
Αν και πολλοί τομείς της Τεχνητής Νοημοσύνης έχουν χρησιμοποιηθεί και συνεχίζουν να χρησιμοποιούνται με μεγάλη επιτυχία στην βιομηχανία των \textit{Video Games}, η Μηχανική Μάθηση δεν έχει την χρήση και δημοτικότητα που θα περιμέναμε απότι είναι γνωστό. Υπάρχουν πολλές θεωρίες σχετικά με αυτό το θέμα όπως θα δούμε. 
\par
Λόγω της μεγάλης ανταγωνιστικότητας και για την προστασία της τεχνολογικής γνώσης, οι εταιρείες του \textit{Game Development} δεν συνηθίζουν να περιγράφουν ή να ανακοινώνουν τα συστήματα τους και με τους αλγορίθμους που υλοποιήθηκαν. Συνεπώς είναι πολύ δύσκολο να γνωρίζουμε με σιγουριά αν και που χρησιμοποιείται Μηχανική Μάθηση στον τομέα του \textit{Game Development}.
\par
Γνωρίζουμε όμως ότι τα παιχνίδια που έχουν ή χρησιμοποιούν Μηχανική Μάθηση είναι ένα πολύ μικρό υποσύνολο των παιχνιδιών που υπάρχουν , συνεπώς επικρατεί η θεωρία ότι ο τομέας της Μηχανικής Μάθησης δεν έχει μεγάλη δημοτικότητα, στον κόσμο του \textit{Game Development}.  Αυτό μπορεί να οφείλεται και στο γεγονός ότι οι μέθοδοι της Μηχανικής Μάθησης δεν μπορούν να εγγυηθούν για τα αποτελέσματα τους ή ότι θα υπάρχουν αποτελέσματα που θα είναι αξιοποιήσιμα στο τελικό προιόν. Αν και υπάρχουν μεθοδολογίες και αλγόριθμοι που είναι αποτελεσματικοί σε ένα μεγάλο έυρος προβλημάτων, η διαδικασία υλοποίησης ενός αλγορίθμου Μηχανικής Μάθησης περιλαμβάνει πολύ πειραματισμό σε κάθε περίπτωση.
\par
Αυτά τα στοιχεία αποθαρρύνουν τις εταιρείες από το να επενδύσουν χρόνο και ανθρώπους σε ένα κομμάτι του παιχνιδιού που μπορεί τελικά να μην είναι αξιοποιήσιμο. Αυτά τα προβλήματα θα λύνονται όσο προχωράει ο τομέας της έρευνας πάνω στο θέμα των εφαρμογών της Μηχανικής Μάθησης σε \textit{Video Games}.

\subsection{Video Games που χρησιμοποιούν Μηχανική Μάθηση}
Θα αναλύσουμε μερικά παιχνίδια που έχουν χρησιμοποιήσει μεθόδους Μηχανικής Μάθησης με επιτυχία σε διάφορους τομείς τους.


\begin{description}

\item[$\bullet$ The Legend of Zelda (1986)] Ένα από τα πιο γνωστά παιχνίδια παγκοσμίως που έγινε σειρά παιχνιδιών είναι το \textit{The Legend of Zelda}. Για το συγκεκριμένο παιχνίδι έγιναν προσπάθειες να συνδυαστούν οι μέθοδοι \textit{Bayesian Network} και \textit{PCA} για να δημιουργηθούν επίπεδα που να μοιάζουν με τα υπάρχον επίπεδα του παιχνιδιού.

\item[$\bullet$ Black \& White (2001)] To Black \& White είναι ένα \textit{Strategy} παιχνίδι που δημοσιεύτηκε το 2001, και έγινε αμέσως επιτυχία χάρη στο πρωτοποριακό AI του για το οποίο βραβεύτηκε παγκοσμίως. Το παιχνίδι χρησιμοποιεί \textit{Decision Trees} και \textit{Reinforcement Learning}, δύο τεχνικές Μηχανικής Μάθησης για να δώσει την δυνατότητα τον παίκτη να εκπαιδεύσει την συμπεριφορά του \textit{NPC} χαρακτήρα, που έχει μορφή ενός γιγάντιου ζώου. 

\item[$\bullet$ Galactic Arms Race (2010)] To Galactic Arms Race (GAR) είναι ένα \textit{Space Shooter} και \textit{Action RPG} παιχνίδι. Χρησιμοποιεί μεθόδους \textit{Neuroevolution} σε συνεργασία με μεθόδους \textit{PCG} για την δημιουργία μοναδικών όπλων για τον κάθε χρήστη ανάλογα με το πως παίζει. 

\end{description}

\par
Όπως βλέπουμε η λίστα που αναλύετε εδώ είναι ιδιαίτερα μικρή. Ευελπιστούμε ότι με εργασίες σαν αυτή θα βοηθήσουμε να ανοίξει ο δρόμος για την χρήση Μηχανικής Μάθησης σε περισσότερα παιχνίδια από εδώ και πέρα.


\section{Ορολογία Μηχανικής Μάθησης}
Θα ορίσουμε μερικές έννοιες από την ορολογία της Μηχανικής Μάθησης για την καλύτερη κατανόηση της υλοποίησης αυτής της εργασίας.Αυτές οι έννοιες θα χρησιμοποιηθούν στην συνέχεια για να περιγράψουμε το κομμάτι του \textit{MLCG}.

\begin{description}

\item[$\bullet$ Sample] Ένα δείγμα εκπαίδευσης. Αποτελεί ένα μεμονομένο δείγμα του σύνολου των πληροφοριών που έχουμε για την εκπαίδευση. Μπορεί να είναι μια σειρά σε ένα αρχείο με δεδομένα, μια εικόνα, ένα βίντεο, ένα ολόκληρο κείμενο ή οτιδήποτε άλλο αναπαριστά ένα μεμονομένο δείγμα στην υλοποίηση μας.

\item[$\bullet$ Dataset] Όλα τα δεδομένα που έχουμε στη διάθεση μας για την διαδικασία της εκπαίδευσης και του ελέγχου. Το \textit{dataset} συνιθίζετε να χωρίζετε σε \textit{training samples} και \textit{testing samples}. 

\item[$\bullet$ Training samples] Ένα υποσύνολο του \textit{Dataset}. Αυτά είναι τα \textit{samples} που χρησιμοποιούνται για την εκπαίδευση του μοντέλου. 

\item[$\bullet$ Testing samples] Ένα υποσύνολο του \textit{Dataset}. Αυτά είναι τα \textit{samples} που χρησιμοποιούνται για τον έλεγχο και την αξιολόγηση του εκπαιδευμένου μοντέλου. 

\item[$\bullet$ Μοντέλο] Είναι μια μέθοδος μηχανικής μάθησης που εκπαιδεύουμε χρησιμοποιώντας το \textit{training sample} ώστε να εφαρμοστεί σε άγνωστα δεδομένα (\textit{testing sample}). Αναφέρονται και ως μοντέλα πρόβλεψης καθώς μετά την εκπαίδευση, προβλέπουν το αποτέλεσμα με βάση τα δεδομένα εισόδου. 

\end{description}

\section{Κατηγορίες Μηχανικής Μάθησης}
Τις μεθόδους της Μηχανικής Μάθησης μπορούμε να τις κατηγοριοποιήσουμε ανάλογα με το είδος της εκπαίδευσης και τις μεθοδολογίες σε συγκεκριμένες κατηγορίες. Η κάθε κατηγορία αποτελεί από μόνη της ένα ολόκληρο επιστημονικό πεδίο, με τις ιδιαιτερότητες και τα προβλήματα στα οποία εξειδικεύεται να το χαρακτηρίζουν. Θα περιγράψουμε μερικές από τις πιο μεγάλες και σημαντικές κατηγορορίες της Μηχανικής Μάθησης.

\begin{description}

\item[$\bullet$ Supervised learning] Οι μέθοδοι αυτής της κατηγορίας δημιουργούν μοντέλα που αντιστοιχούν σε μαθηματικές αναπαραστάσεις των δεδομένων. Η εκπαίδευση γίνετε με δεδομένα που γνωρίζουμε το αποτέλεσμα που επιθυμούμε να μας δώσουν. Μέσα από διάφορες μεθόδους, η Μηχανική Μάθηση προσπαθεί να βρει μια συνάρτηση που να εκφράζει καλύτερα τα δεδομένα εκπαίδευσης. Έχοντας αυτή την συνάρτηση μπορεί να πάρει το επιθυμητό αποτέλεσμα για άγνωστα δεδομένα.

\item[$\bullet$ Unsupervised learning] Στην κατηγορία του \textit{Unsupervised learning} δεν γνωρίζουμε το αποτέλεσμα ή χαρακτηριστικό όπως στο \textit{Supervised learning}. Σε αυτή την κατηγορία προσπαθούμε να βρούμε μοτίβα και κοινά χαρακτηριστικά μεταξύ των δειγμάτων μας ώστε να τα κατηγοροιοποιήσουμε σε ομάδες. Στη συνέχεια εφαρμόζοντας την ίδια κατηγοριοποίηση σε άγνωστα δεδομένα να βρούμε στοιχεία από τις ομάδες στις οποίες ανήκουν.

\item[$\bullet$ Reinforcement learning] Το \textit{Reinforcement learning} έχει μεγάλη συσχέτιση με τον τρόπο που μαθαίνουν οι άνθρωποι. Χρησιμοποιεί μεθόδους ανταμοιβής και τιμωρίας για να αποδώσουν ή να αφαιρέσουν βαθμούς από το μοντέλο. Σε αυτές τις κατηγορίες το μοντέλο έχει ως στόχο να μεγιστοποιήσει την ανταμοιβή του ή να ελαχιστοποιήσει την τιμωρία του, οπότε εφαρμόζει \textit{search algorithms} για να αλληλεπιδράσει με το περιβάλλον/σύστημα και να βρει για κάθε περίπτωση ποια έιναι η καλύτερη δυνατή στρατηγική ώστε να πετύχει τον στόχο του.

\end{description}

Η κατηγορία που χρησιμοποιήθηκε στη συγκεκριμένη εργασία ανήκει στο \textit{Supervised learning} καθώς για κάθε δεδομένα εκπαίδευσης υπήρχε το αντίστοιχο δεδομένο-αποτέλεσμα που επιθυμούσαμε να επιστρέψει το σύστημα.

\subsection{Μέθοδοι Supervised learning}
Θα αναλύσουμε μερικές μεθόδους της κατηγορίας \textit{Supervised learning} καθώς αυτή χρησιμοποιήθηκε για την υλοποίηση αυτής της εργασίας.

\begin{description}

\item[$\bullet$ Decision trees] Τα \textit{Decision Trees} όπως περιγράφει και το όνομα τους είναι μοντέλα πρόβλεψης με δενδρική αναπαράσταση. Χρησιμοποιούν τα χαρακτηριστικά των δεδομένων εισόδου για να περιηγηθούν μέσα στο δέντρο μέχρι να φτάσουν σε κάποιο αποτέλεσμα-ρίζα. Κατά την εκπαίδευση του μοντέλου τα χαρακτηριστικά των δεδομένων εκπαίδευσης χωρίζονται με συγκεκριμένες μεθόδους, ανάλογα τον αλγόριθμο του \textit{Decision Tree} για να δημιουργήσουν το τελικό μοντέλο-δέντρο.

\item[$\bullet$ Support vector machines] Ένα \textit{SVM} στην ποιο απλή του μορφή, αποτελεί μια μαθηματική μέθοδο διαχωρισμού των δεδομένων σε δύο ομάδες. Χρησιμοποιεί μεθόδους διχοτόμησης του χώρου των δεδομένων ώστε τα στοιχεία της κάθε ομάδας να είναι στον ίδιο χώρο. Έχει τη δυνατότητα να ανάγει τις μεθόδους της σε δεδομένα πολύδιάσταστατων χώρων, ένα πολύ χρήσιμο χαρακτηριστικό ειδικά σε δύσκολα προβλήματα κατηγοριοποίησης. 

\item[$\bullet$ Artificial neural networks] Τα \textit{Artificial neural networks} είναι συστήματα που έχουν φτιαχτεί με έμνευση τον ανρθώπινο εγκέφαλο και τον τρόπο λειτουργίας του. Αποτελούνται από \textit{Artificial Neurons} οι οποίοι παίρνουν μια είσοδο, την επεξεργάζονται και βγάζουν ένα αποτέλεσμα που μπορεί να είναι είσοδος για τον επόμενο \textit{Artificial Neuron}. Αυτά τα στοιχεία είναι οργανωμένα σε επίπεδα (\textit{layers}) και η έξοδος του κάθε επιπέδου αποτελεί είσοδο για το επόμενο. \textit{Artificial neural networks} μιας αρχιτεκτονικής που ονομάζετε \textit{Generative Adversarial Networks} χρησιμοποιήθηκε για την υλοποίηση του MLCG αυτής της εργασίας.

\end{description}



\section{Artificial Neural Networks (ANN)}
Τα ANNs έχουν γνωρίσει μεγάλη δημοσιότητα τα τελευταία χρόνια χάρη στα πρωτοποριακά αποτελέσματα που παρουσιάζουν σε πολλά προβλήματα τεχνητής νοημοσύνης, όπως την επεξεργασία και ανάλυση εικόνας και βίντεο. Η επιτυχία τους οφείλετε σε μεγάλο βαθμό στην πρόοδο του hardware των ηλεκτρονικών υπολογιστών το οποίο έδωσε τους απαραίτητους πόρους για την ανάπτυξη των λεγόμενων Deep Neural Networks. Αν και η θεωρία και η έρευνα πάνω στα ANNs ξεκίνησε το 1940, οι δυνατότητες τους ξεκίνησαν να γίνονται αντιληπτές στο διάστημα 2009 με 2012 όπου υλοποιήσεις με ANNs ξεκίνησαν ξεπερνάνε τα μέχρι τότε καλύτερα μοντέλα μηχανικής μάθησης σε διάφορους τομείς, και να προσεγγίζουν τα ανθρώπινα όρια.

\subsection{Χαρακτηριστικά ενός ANN}
Θα αναλύσουμε μερικά από τα πιο βασικά χαρακτηριστικά ενός ANN τα οποία περιέχονται σε κάθε διαφορετική υλοποίηση των ANNs

\begin{description}

\item[$\bullet$ Neuron] Ένα ANN αποτελείται από Artificial Neurons (Τεχνητοί Νευρώνες) σε αντιστοιχία με έναν ανθρώπινο εγκέφαλο. Ο κάθε νευρώνας δέχετε μια ή περισσότερες είσοδους, οι οποίες περνάνε από την συνάρτηση ενεργοποίησης του (activation function) μαζί με ένα βάρος. Η έξοδος της συνάρτησης ενεργοποίησης περνάει στην συνάρτηση εξόδου που είναι και το αποτέλεσμα που επιστρέφει ο νευρώνας ως έξοδο.

\item[$\bullet$ Activation function] Η συνάρτηση ενεργοποίησης έχει ως σκοπό να εναρμονίσει την έξοδο του νευρώνα με την είσοδο που δέχετε, ώστε μικρές αλλαγές στην είσοδο να επιφέρουν και μικρές αλλαγές στην έξοδο. Λειτουργεί και ως διακόπτης του νευρώνα, ο οποίος για συγκεκριμένες τιμές παράγει μηδενική έξοδο.

\item[$\bullet$ Weights] Τα βάρη είναι τιμές που αποδίδονται σε κάθε είσοδο ενός νευρώνα. Λαμβάνονται υπόψην μαζί με την τιμή της εισόδου για να παράξουν την επιθυμητή τιμή εξόδου. Τα βάρη είναι από τα πιο σημαντικά κομμάτια ενός νευρωνικού δικτύου καθώς αυτές οι τιμές είναι που αλλάζουν κατα την εκπαίδευση για να φτιαχτεί ένα νευρωνικό που να παράγει τα αποτελέσματα που θέλουμε.

\item[$\bullet$ Connections] Οι συνδέσεις μεταξύ νευρώνων είναι επίσης πολύ σημαντικές και είναι ένα από τα κομμάτια που καθορίζουν την αρχιτεκτονική του δικτύου. Οι συνδέσεις δεν αλλάζουν κατά την εκπαίδευση αλλά μπορούν να "αποκοπούν" προσωρινά για να βελτιωθεί η εκπαίδευση.

\item[$\bullet$ Layers] Τα ANNs είναι οργανωμένα σε επίπεδα, όπου το κάθε επίπεδο έχει ένα συγκέκριμένο αριθμό από νευρώνες, δέχετε συγκεκριμένο αριθμό από εισόδους σε αυτούς τους νευρώνες και βγάζει συγκεκριμένο αριθμό εξόδων για το επόμενο επίπεδο ή ως τελικό αποτέλεσμα. Το πρώτο επίπεδο ενός ANN ονομάζετε επίπεδο εισόδου (input layer) ενώ αντίστοιχα το τελευταίο επίπεδο ονομάζετε επίπεδο εξόδου (output layer). Υπάρχουν πολλών ειδών επίπεδα ανάλογα με τον αριθμό των νευρώνων και τις συνδέσεις τους με προηγούμενα και επόμενα επίπεδα. 

\end{description}


\subsection{Επίπεδα}
Θα αναλύσουμε μερικά από τα πιο γνωστά επίπεδα καθώς και αυτά που χρησιμοποιήθηκαν σε αυτή την υλοποίηση. Η σχεδίαση της αρχιτεκτονικής ενός ANN περιλαμβάνει την επιλογή των επιπέδων που θα περιέχει, την σύνδεση τους και την ιεραρχία τους καθώς και τον αριθμό των εισόδων τους. Υπάρχουν μεθοδολογίες για την επιλογή επιπέδων και σχεδίαση αρχιτεκτονικών αλλά τα περισσότερα μοντέλα δημιουργούνται με συνεχείς επαναλήψεις δοκιμών και αναδιοργάνωσης μέχρι να βρεθεί η αρχιτεκτονική με τα καλύτερα αποτελέσματα.

\begin{description}

\item[$\bullet$ ReLU] Rectified Linear Unit είναι ολόκληρο το όνομα αυτού του επιπέδου, για συντομία ReLU. Είναι ένα γραμμικό επίπεδο με συνάρτηση ενεργοποίησης και εξόδου την $f(x) = max(x, 0)$. Επίπεδο για τον μηδενισμό όλων των αρνητικών τιμών.

\item[$\bullet$ BatchNormalization] Όπως περιέχετε και στο όνομα, αυτό το επίπεδο εφαρμόζει μια συνάρτηση κανονικοποίησης στα δεδομένα ώστε να διατηρούνται οι τιμές από το ένα επίπεδο στο άλλον μέσα σε ένα συγκεκριμένο εύρος. Το επίπεδο BatchNormalization μπορεί να εφαρμοστεί στα δεδομένα εισόδου ή εξόδου ανάμεσα σε άλλα επίπεδα, και κρατάει τη μέση τιμή ενεργοποίησης κοντά στο 0 και την τυπική απόκλιση κοντά στο 1.

\item[$\bullet$ Dropout ] Το Dropout επίπεδο χρησιμοποιείται μόνο κατά την διάρκεια της εκπαίδευσης. Αυτό το επίπεδο αλλάζει τυχαία με βάση κάποια παράμετρο, κάποιες από τις τιμές εισόδου σε 0. Αυτό έχει ως αποτέλεσμα την "απενεργοποίηση" των νευρώνων που λαμβανουν αυτές τις εισόδους. Ο αριθμός των εισόδων που θα γίνουν 0 ορίζεται ως παράμετρος που επιπέδου, για παράδειγμα με παράμετρο 0.2, το 20\% των εισοδων θα γίνει 0. Το επίπεδο Dropout χρησιμοποιείται για την αποφυγή του overfitting.

\item[$\bullet$ Dense] Το Dense είναι ένα μη γραμμικό, βαθύ επίπεδο που μπορεί εν δυνάμει να αναπαραστήσει οποιαδήποτε μαθηματική συνάρτηση. Είναι από τα πιο δημοφιλή επίπεδα των Deep ANN. Έχουν για συνάρτηση εξόδου  $f(x) = activation((x \bullet kernel) + bias)$. Όπου ο kernel αντιστοιχεί στα weights της εισόδου τα οποία είναι σε διανυσματική μορφή, το x είναι η είσοδος του νευρώνα επίσης σε διανυσματική μορφή, το bias είναι μια σταθερή τιμή και η activation είναι η συνάρτηση ενεργοποίησης.

\item[$\bullet$ Conv2D] Από τα πιο ευρέως διαδεδομένα επίπεδα είναι τα Convolution. Με την χρήση ενός convolutional matrix εφαρμόζει συνέλιξη στα δεδομένα εισόδου για να παράξει την έξοδο. Τα δεδομένα εισόδου πρέπει να είναι σε διανυσματική μορφή.

\end{description}



\subsection{Backpropagation}
Ένας από τους πιο σημαντικούς αλγορίθμους για την λειτουργία των Deep ANNs είναι ο αλγόριθμος του Backpropagation. Όπως δηλώνει και το όνομα του, ο Backpropagation είναι η μέθοδος που "αναθέτει" το σφάλμα από ένα πέρασμα του δικτύου, στα βάρη των νευρώνων ώστε να είναι ισοκατανεμημένο ανάλογα με την συμμετοχή του κάθε νευρώνα στην τελική έξοδο. Στη συνέχεια τα βάρη διορθώνονται με βάση το σφάλμα που τους αναλογεί και η εκπαίδευση συνεχίζετε. Με κάθε είσοδο ή σε batches εισόδων, επαναλαμβάνετε αυτή η διαδικασία μέχρι τα βάρη να μην έχουν μεγάλες μεταβολές σε κάθε ενημέρωση.

\par
Η μέθοδος του  Backpropagation είναι αυτή που έκανε δυνατή την ύπαρξη των Deep ANNs καθώς αντιμετωπίζει το πρόβλημα των vanishing gradient problem όπου το λάθος που επιστρεφόταν από τα τελικά επίπεδα προς τα αρχικά γινόταν 0 μετά από ορισμένο αριθμό επιπέδων και το δίκτυο δεν μπορουσε να εκπαιδευτεί.




\section{Generative Adversarial Networks (GANs)}
Τα Generative Adversarial Networks αποτελούν μια κατηγορία των ANNs η οποία δημιουργήθηκε το 2014 από τον Ian Goodfellow και τους συνεργάτες του. Η βασική ιδέα πίσω από τα GANs περιλαμβάνει δύο ANN τα οποία λειτουργούν ανταγωνιστικά το ένα με το άλλο. Λέγονται Generative επειδή ως στόχο τα GANs έχουν την δημιουργία περιεχομένου. Αρχικά τα GANs είχαν προταθεί ως μέθοδοι για Unsupervised εκπαίδευση αλλά στη συνέχεια επεκτάθηκαν και υπάρχουν υλοποιήσεις τους για Supervised, Semi-supervised και Reinforcement Μηχανική Μάθηση. 
\par
Βασικός στόχος ενός GAN είναι η παραγωγή περιεχομένου που να μοιάζει όσο το δυνατόν πιο αληθινό. Για την εκπαίδευση του χρειάζετε ένα dataset με πραγματικά παραδείγματα από περιεχόμενο όπως αυτό που θέλουμε να παράγει το δίκτυο, όπως εικόνες ανθρώπων ή επίπεδα ενός παιχνιδιού (όπως συμβαίνει στην συγκεκριμένη υλοποίηση).

\subsection{Μοντέλο GAN}
Όπως αναφέρθηκε, ένα GAN αποτελείται από δύο ΑΝΝs, το Generative Network και το Discriminative Network. Το κάθε ANN έχει την δική του αρχιτεκτονική και παραμέτρους. Τα δύο δίκτυα εκπαιδεύονται μέσω ανταγωνισμού καθώς η έξοδος του ενός αποτελεί την είσοδο του άλλου και συνεχώς προσπαθούν να ξεπεράσουν το ένα το άλλο.


\begin{description}

\item[$\bullet$ Generative Network] Το Generative Network είναι υπεύθυνο για την παραγωγή του επιθυμητού περιεχομένου με βάση την τιμή της εισόδου που του δίνουμε. Αυτό το περιεχόμενα που θα παραχθεί θα δωθεί ως είσοδο στο Discriminative Network. H έξοδος του Discriminative θα δωθεί πάλι στο Generative ώς feedback για να γίνει διόρθωση, αν χρειάζετε των weights του.

\item[$\bullet$ Discriminative Network] Το Discriminative Network έχει ως στόχο των διαχωρισμό των πραγματικών δειγμάτων από αυτά που παράγει ο Generator. Λαμβάνει ως είσοδο ένα δείγμα και "αποφασίζει" εάν είναι πραγματικό δείγμα ή παραγμένο από τον Generator. Αυτή την απόφαση την δίνει ως feedback στον Generator και αυτό λαμβάνει ως feedback εάν η πρόβλεψη του ήταν σωστή ή όχι για να διορθώσει τα weights του.

\end{description}

\subsection{Εκπαίδευση}
H διαδικασία της εκπαίδευσης περιλαμβάνει την συνεχή ανατροφοδότηση των δύο δικτύων όπου σταδιακά γίνονται καλύτερα, το καθένα στον σκοπό του, καθώς ανταγωνίζονται το ένα το άλλο. Για την εκπαίδευση απαιτείται ένα dataset πραγματικών παραδειγμάτων που δίνονται στο κάθε δίκτυο με διαφορετική επισύμανση ανάλογα με τον σκοπό τους.
\par
Στον Generator δίνετε μια είσοδος και η έξοδος που θα έπρεπε να παράξει από το σύνολο των πραγματικών δειγμάτων. Στον Discriminator δίνετε αντίστοιχα ένα δείγμα, (πραγματικό ή παραγμένο), και η επισήμανση του αντίστοιχα, εαν είναι πραγματικό ή όχι. Η ανατροφοδότηση μεταξύ των δύο δικτύων βελτιώνει συνεχώς τις αποδόσεις και των δύο.


\subsection{Αποτελέσματα και αποδόσεις}
Τα αποτελέσματα ενός GAN όπως και όλων των ΑΝΝ σχετίζονται άμεσα με την αρχιτεκτονική τους. Ο σχεδιασμός και η παραμετροποίηση ενός ANN αποτελεί ένα πολύ δύσκολο έργο το οποίο περιλαμβάνει συνεχόμενες επαναλήψεις και δοκιμές. Αντίστοιχα η σχεδίαση και η εκπαίδευση δύο δικτύων, όπου το ένα αλληλεπιδρά με το άλλο αποτελεί ένα έργο διπλάσιας δυσκολίας με διπλάσιες απαιτήσεις σε πόρους  και δοκιμές. 
\par
Τα αποτελέσματα των GANs δικαιολογούν το κόστος τους καθώς παράγουν ιδιαίτερα μοναδικό και αξιόπιστο περιεχόμενο, δύο χαρακτηριστικά που απαιτούνται στον τομέα του Content Generation όπως αναλύθηκε στο κεφάλαιο 1.



























